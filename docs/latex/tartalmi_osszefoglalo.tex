\chapter*{Tartalmi összefoglaló}
\addcontentsline{toc}{section}{Tartalmi összefoglaló}
\spacing{1.5}
A dolgozat legfőbb célja a webalkalmazás működésének és fejlesztési menetének bemutatása. A feladat egy fraktál bemutató keretrendszer elkészítése volt, amelyben többek között olyan hasznos funkciók találhatók, mint az algoritmusok konfigurálhatósága, az algoritmusok futásának szüneteltetése, az algoritmusok állapotának visszaállítása egy korábbi állapotra, illetve a generált fraktál kiexportálása kép formájában. A dolgozat magába foglal egy bevezető szekciót, ahol a fraktálokról általános tudnivalókat olvashatunk. Itt lényegében a fraktálok eredetéről, fontosabb tulajdonságairól, illetve főbb felhasználási területeiről van szó. Ezen kívül, a dolgozat további részeiben magáról a webalkalmazásról van szó, ismertetem a felhasznált technológiákat, valamint részletezem a fejlesztés menetét.