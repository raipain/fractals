\chapter*{Feladatkiírás}
%A tartalomjegyzékben mégis szerepeltetni kell, mint szakasz(section) szerepeljen:
\addcontentsline{toc}{section}{Feladatkiírás}
\spacing{1.5}
A szakdolgozatom központi témájául a fraktálokat választottam. Matematikában nem túl jártas egyéneknek lehetséges, hogy nem túl sokat mondó az a fogalom, hogy fraktál. Az én célom egy olyan webes applikáció elkészítése volt, amelyet használva mindenki mélyebb belátást nyerhet a fraktálok világába. A webes applikációban nyolc előre megírt algoritmus szerint generálhatunk fraktálokat. A felhasználóknak lehetőségük van az algoritmusok testreszabására, minden algoritmus egyéni konfigurálási lehetőségekkel rendelkezik. Ilyenek például az elforgatási szög módosítása, tetszőleges méretű, pozíciójú, valamint tetszőlegesen rotálható gyökér elem megadása, növekedés irányának módosítása, szín beállítása... Az applikáció arra is lehetőséget biztosít, hogy valamely időpontban az algoritmus futását szüneteljük, valamint ha szeretnénk, egy csúszka használatával visszatekerhetjük az algoritmus pillanatnyi állapotát valamely előző állapotra. Ezen kívül az alkalmazásban egy olyan funkció is található, amely segítségével a felhasználók a fraktálok generálása az adott fraktálról egy rövid leírást olvashatnak, ezzel is lehetőséget adva a felhasználóknak, hogy tudásukat bővítsék. Végül a generált fraktálokat kép formájában kiexportálhatjuk a saját gépünkre, ha szeretnénk.

