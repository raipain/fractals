\chapter{Háttér}
\spacing{1.5}

Az alkalmazás elkészítése előtt kiválasztottam azokat a fraktálgeneráló algoritmusokat, amelyeket implementálni szerettem volna. Nyolc algoritmus mellett döntöttem, ezek a következők: 
\begin{itemize}
	\item Sierpiński- háromszög
	\item Sierpiński-szőnyeg
	\item Fraktál-fa
	\item Pitagorasz-fa
	\item Lévy C-görbe
	\item Koch-görbe
	\item Hilbert-görbe
	\item H-fa
\end{itemize}
Választáskor többek között figyelembe vettem az algoritmusok bonyolultságát, milyen, és mennyi konfigurációs lehetőséggel tudom felruházni őket, valamint a fraktálok fajtáját. A választott algoritmusok között vannak helykitöltő algoritmusok is, mint például a Hilbert-görbe és a H-fa. Ezek az algoritmusok nagyjából kivétel nélkül már mások által implementálva vannak, a forráskódjuk bárki számára elérhető az interneten, viszont ezek az implementációk egyike sem támogatja a konfigurálhatóságot, ami az én alkalmazásom egyik alapköve. 
\par Mielőtt az implementációnak nekikezdtem, utánanéztem olyan webes alkalmazásoknak, amik hasonlóak, mint amit én is szerettem volna megvalósítani. Találtam hasonlót, ahol némelyik fraktálgeneráló algoritmus valamilyen szinten konfigurálható, viszont ezek az alkalmazások jellemzően csak egy-egy fraktállal foglalkoznak különösebben, így csak az adott fraktál generálására van lehetőségünk, valamint kevesebb konfigurálási lehetőséggel rendelkeznek, mint az én tervezett webalkalmazásom. Ezen kívül nem támogatnak kiexportálási és visszatekerési funkciókat. Az algoritmusok konfigurálása ezekben az alkalmazásokban egyszerű, jelölőnégyzettel, vagy számok bevitelére alkalmas mezőkkel történik, hasonlóan, mint ahogyan én is terveztem. 
\par Az interneten való keresgélést követően kiválasztottam azt a technológiát, amellyel magát a keretrendszert szerettem volna implementálni.
Mivel ez egy webes alkalmazás, a választás az Angular keretrendszerre esett, pontosabban annak a 8-as verziójára~\cite{angular}. Ez nem a legfrissebb verziója az Angular keretrendszernek, viszont korábban is használtam, és stabilitás szempontjából ez az egyik legjobb, ugyanis vannak olyan különböző API-k, amelyek még nem teljesen kompatibilisek az Angular legújabb verziójával. A fontosabb Angular API-k, amiket használtam a fejlesztés során a következők: Angular Material, Angular Flex Layout és Rxjs. Az Angular egy TypeScript keretrendszer, amellyel, a benne található csomagok segítségével rendkívül kényelmesen és könnyen készíthetünk Single Page webalkalmazásokat. Ezen kívül a JavaScript-es könyvtárak nagy részével kompatibilis, ami nagyon előnyös volt számomra, hiszen az algoritmusokat P5.js-ben implementáltam, ami egy nyílt forráskódú, vászon alapú rajzolásra alkalmas JavaScript könyvtár~\cite{p5}. Az alapok nagyon könnyen elsajátíthatóak, és a közismert alakzatok nagy részének rajzolása előre megírt függvények segítségével történik, ami rendkívül kényelmessé teszi a használatát. Ezen kívül vektorok használatát is támogatja, a vektorműveletek már előre definiálva vannak benne, amelyeket az algoritmusok implementálásakor előszeretettel használtam. Egyik funkcionalitása, ami még nagyon sokat segített az algoritmusok implementálásakor, az a $translate$ függvény, amely segítségével a vászon koordinátarendszerének origóját tudjuk eltolni. Ez egy rendkívül hasznos metódus, különböző rotációknál, és olyan alakzatok rajzolásánál, amelyek pozíciója az egérmutató elhelyezkedésétől függ. Ezek a műveletek az előre megírt függvényeknek köszönhetően sokkal kevesebb számolással kivitelezhetőek.
\par
A fejlesztés Windows 10 operációs rendszer alatt történt. A programozáshoz, futtatáshoz, teszteléshez a Visual Studio Code fejlesztői környezetet használtam. A szoftver verziókövetése, illetve a fájlok tárolása GitHub repository-n keresztül valósult meg.
