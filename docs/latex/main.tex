\documentclass[12pt]{report}

\usepackage[utf8]{inputenc}
\usepackage[T1]{fontenc}
\usepackage[magyar]{babel}

\usepackage{times}

\usepackage{amsmath}
\usepackage{amssymb}
\usepackage{amsthm}

\usepackage{fancyhdr}

\usepackage{graphicx}
\usepackage{psfrag}

\usepackage{listings}

\usepackage{url}

\usepackage{setspace}

\usepackage{enumitem}
\usepackage{todonotes}

\newtheorem{tét}{Tétel}[chapter]
\newtheorem{defi}[tét]{Definíció}
\newtheorem{lemma}[tét]{Lemma}
\newtheorem{áll}[tét]{Állítás}
\newtheorem{köv}[tét]{Következmény}

\theoremstyle{definition}
\newtheorem{megj}[tét]{Megjegyzés}
\newtheorem{pld}[tét]{Példa}

%Margók:
\hoffset -1in
\voffset -1in
\oddsidemargin 35mm
\textwidth 150mm
\topmargin 15mm
\headheight 10mm
\headsep 5mm
\textheight 237mm

% Listing beállítások
\lstset{basicstyle=\small,  showstringspaces=false, breaklines=true}
\renewcommand{\lstlistingname}{Kódrészlet}


\begin{document}

%A FEJEZETEK KEZDÕOLDALAINAK FEJ ES LÁBLÉCE:
%a plain oldalstílust kell átdefiniálni, hogy ott ne legyen fejléc:
\fancypagestyle{plain}{%
%ez mindent töröl:
\fancyhf{}
% a láblécbe jobboldalra kerüljön az oldalszám:
\fancyfoot[R]{\thepage}
%elválasztó vonal sem kell:
\renewcommand{\headrulewidth}{0pt}
}

%A TÖBBI OLDAL FEJ ÉS LÁBLÉCE:
\pagestyle{fancy}
\fancyhf{}
\fancyhead[L]{Fraktál bemutató keretrendszer}
\fancyfoot[R]{\thepage}


%A címoldalra se fej- se lábléc nem kell:
\thispagestyle{empty}

\begin{center}
\vspace*{1cm}
{\Large\bf Szegedi Tudományegyetem}

\vspace{0.5cm}

{\Large\bf Informatikai Intézet}

\vspace*{3.8cm}


{\LARGE\bf Fraktál bemutató keretrendszer }


\vspace*{3.6cm}

{\Large Szakdolgozat}

\vspace*{4cm}

%Értelemszerûen megváltoztatandó:
{\large
\begin{tabular}{c@{\hspace{4cm}}c}
\emph{Készítette:}     &\emph{Témavezetõ:}\\
\bf{Péter Albert}  &\bf{Tóth Zotán Gábor}\\
programtervező informatikus     &egyetemi tanársegéd\\
szakos hallgató&
\end{tabular}
}

\vspace*{2.3cm}

{\Large
Szeged
\\
\vspace{2mm}
\the\year
}
\end{center}


%A tartalomjegyzék:
\tableofcontents


\chapter*{Feladatkiírás}
%A tartalomjegyzékben mégis szerepeltetni kell, mint szakasz(section) szerepeljen:
\addcontentsline{toc}{section}{Feladatkiírás}
\spacing{1.5}
A szakdolgozatom központi témájául a fraktálokat választottam. Matematikában nem túl jártas egyéneknek lehetséges, hogy nem túl sokat mondó az a fogalom, hogy fraktál. Az én célom egy olyan webes applikáció elkészítése volt, amelyet használva mindenki mélyebb belátást nyerhet a fraktálok világába. A webes applikációban nyolc előre megírt algoritmus szerint generálhatunk fraktálokat. A felhasználóknak lehetőségük van az algoritmusok testreszabására, minden algoritmus egyéni konfigurálási lehetőségekkel rendelkezik. Ilyenek például az elforgatási szög módosítása, tetszőleges méretű, pozíciójú, valamint tetszőlegesen rotálható gyökér elem megadása, növekedés irányának módosítása, szín beállítása... Az applikáció arra is lehetőséget biztosít, hogy valamely időpontban az algoritmus futását szüneteljük, valamint ha szeretnénk, egy csúszka használatával visszatekerhetjük az algoritmus pillanatnyi állapotát valamely előző állapotra. Ezen kívül az alkalmazásban egy olyan funkció is található, amely segítségével a felhasználók a fraktálok generálása az adott fraktálról egy rövid leírást olvashatnak, ezzel is lehetőséget adva a felhasználóknak, hogy tudásukat bővítsék. Végül a generált fraktálokat kép formájában kiexportálhatjuk a saját gépünkre, ha szeretnénk.


\chapter*{Tartalmi összefoglaló}
\addcontentsline{toc}{section}{Tartalmi összefoglaló}
\spacing{1.5}
A dolgozat legfőbb célja a webalkalmazás munkamenetének leírása. A feladat egy fraktál bemutató keretrendszer elkészítése volt, amelyben többek között olyan hasznos funkciók találhatók, mint az algoritmusok konfigurálhatósága, az algoritmusok futásának szüneteltetése, az algoritmus állapotának visszaállítása egy korábbi állapotra, illetve a generált fraktál kiexportálása kép formájában. A dolgozat magába foglal egy bevezető szekciót, ahol a fraktálokról általános tudnivalókat olvashatunk. Itt lényegében a fraktálok eredetéről, fontosabb tulajdonságairól, illetve főbb felhasználási területeiről van szó. Ezen kívül, a dolgozat további részeiben a magáról webalkalmazásról van szó, ismertetem a felhasznált technológiákat, valamint részletezem a fejlesztés menetét.
\chapter{Bevezetés}
%\addcontentsline{toc}{section}{Bevezetés}
\spacing{1.5}

\section{A fraktálokról átalánosságban}
A fraktálok egy viszonylag újnak nevezhető terület a matematikában, és habár komolyabban vizsgálni őket csak a számítógépek megjelenése után kezdték el igazán, már a XX. század előtt is foglalkoztak a matematikusok ezekkel a különös geometriai alakzatokkal. Az elnevezést 1975-ben Benoît Mandelbrot adta, a latin fractus (vagyis törött, törés) szó alapján, ami az ilyen alakzatok tört számú dimenziójára utal, bár nem minden fraktál tört dimenziós (ilyenek például a síkkitöltő görbék)~\cite{fraktal} . 
Tört dimenziósnak nevezik azokat az alakzatokat, amelyek nincs területük, többek mint egy egyenes, viszont kevesebbek mint egy síkidom. Nagyszerű példa erre a Sierpiński-szőnyeg. Mandelbrot {\it The Fractal Geometry of Nature (A Természet Fraktálgeometriája)} című munkája mutatta be, és magyarázta el először a fogalmakat, melyek alapjául szolgálnak ennek az új területnek~\cite{fraktal-wiki}. 
\par A fraktálok önhasonló, végtelenül komplex matematikai alakzatok, melyek változatos formáiban legalább egy felismerhető ismétlődés tapasztalható. Az önhasonlóság azt jelenti, hogy egy kisebb rész felnagyítva ugyanolyan struktúrát mutat, mint egy nagyobb rész. Ilyen bizonyos léptékig például a természetben a villám mintázata, a levél erezete, a hópelyhek alakja, a fa ágai. A fraktál szóval rendszerint az önhasonló alakzatok közül azokra utalnak, amelyeket egy matematikai formulával le lehet írni, vagy meg lehet alkotni~\cite{fraktal}. A fraktálok egy másik tulajdonsága, hogy sehol sem differenciálhatók. Ennek oka az, hogy bár a fraktálok folytonosak, végtelenül gyűröttek. A fraktálokat úgy írhatjuk le, hogy megvizsgáljuk, hogyan változnak különböző felbontásoknál.

\section{A fraktálok felhasználási területei}
A fraktálok gyakorlati felhasználásának köre folyamatosan bővül. Rengeteg helyen találkozhatunk velük a művészetektől az orvosláson át a számítástechnikáig. Ma már bizonyított tény, hogy fraktálok, illetve fraktálszerű alakzatok a természetben is gyakran előfordulnak. Megfigyelések alapján tudjuk, hogy ezek az alakzatok egy meghatározott növekedési struktúrát követnek. Ilyenek a fák, kristályok, vagy a felhők. Számtalan művész használta a fraktálok ábrázolásának technikáját, annak ellenére, hogy pontos definícióját, matematikai hátterét nem ismerték.
Olyan neveket érdemes itt megemlíteni, mint például Vincent Van Gogh, vagy akár Maurits Cornelis Escher. A építészetben is sok önismétlődő részletet találhatunk gondoljunk csak a gótikus és barokk épületek ismétlődő támpilléreire, oszlopsoraira~\cite{alkalmazas}.
\par A fraktálok kiválóan alkalmazhatóak képi és hanganyag tömörítésére, feldolgozására. Ha egy hangot vagy képet fraktálokra bontunk, utána az adott darabot leíró algoritmussal és paramétereivel könnyen összehasonlíthatóvá válnak. Továbbá az algoritmusok és a hozzá tartozó paraméterek letárolása kevesebb területet vesz igénybe, mintha a nyers adatokat tárolnánk le~\cite{alkalmazas}.
\chapter{Háttér}
\spacing{1.5}

Az alkalmazás elkészítése előtt kiválasztottam azokat a fraktál generáló algoritmusokat, amelyeket implementálni szerettem volna. Nyolc algoritmust választottam, ezek a következők: 
\begin{itemize}
	\item Sierpiński- háromszög
	\item Sierpiński-szőnyeg
	\item Fraktál-fa
	\item Pitagorasz-fa
	\item Lévy C-görbe
	\item Koch-görbe
	\item Hilbert-görbe
	\item H-fa
\end{itemize}
Választáskor többek között figyelembe vettem az algoritmusok bonyolultságát, milyen, és mennyi konfigurációs lehetőséggel tudom felruházni az adott algoritmust, valamint a fraktálok fajtáját. A választott algoritmusok között vannak helykitöltő algoritmusok is, mint például a Hilbert-görbe és a H-fa. Ezek az algoritmusok nagyjából kivétel nélkül már mások által implementálva vannak, a forráskódjuk bárki számára elérhető az interneten, viszont ezek az algoritmusok egyike sem támogatja a konfigurálhatóságot, ami az én alkalmazásom egyik alapköve. 
\par Az implementáció elkezdése előtt utánanéztem olyan webes alkalmazásoknak, amik hasonlóak, mint az én tervezett alkalmazásom. Találtam hasonlót, ahol egy-egy fraktál generáló algoritmus valamilyen szinten konfigurálható, viszont ezeken az alkalmazások jellemzően csak egy-egy fraktállal foglalkoznak különösebben, így csak az adott fraktál generálására van lehetőségünk, és kevesebb konfigurálási lehetőséggel rendekeznek, ha egyáltalán rendelkeznek, mint az én tervezett webalakalmazásom, valamint nem támogatnak kiexportálási és visszatekerési lehetőséget. Az algoritmusok konfigurálása ezekben az alkalmazásokban egyszerű, jelölönégyzettel, vagy szám típusú beviteli mezőkkel történik, hasonlóan, mint ahogyan én is terveztem. 
\par Az interneten való keresgélést követően kiválasztottam azt a technológiát, amellyel magát a keretrendszert szerettem volna implementálni.
Mivel ez egy webes alkalmazás, a választás az Angular keretrendszerre esett, pontosabban az Angular 8-as verziójára~\cite{angular}. Ez nem a legfrissebb verziója az Angular keretrendszernek, viszont korábban is használtam, és stabilitás szempontjából ez az egyik legjobb, ugyanis vannak olyan különböző API-k, amelyek még nem teljesen kompatibilisek az Angular legújabb verziójával. A fontosabb Angular API-k, amiket használtam a fejlesztés során a következők: Angular Material, Angular Flex Layout és Rxjs. Az Angular egy TypeScript keretrendszer, amellyel, a benne található csomagok segítségével, rendkívül kényelmesen és könnyen készíthetünk Single Page webalkalmazásokat. Ezen kívül a JavaScript-es könyvtárak nagyon nagy részével kompatibilis, ami nagyon nagy előny volt számomra, hiszen az algoritmusokat P5.js-ben~\cite{p5} implementáltam, ami egy nyílt forráskódú, vászon alapú rajzolásra alkalmas JavaScript könyvtár. Az alapok nagyon könnyen elsajátíthatóak, és a közismert alakzatok nagy részének a rajzolása előre definiált függvények segítségével történik, ami rendkívül kényelmessé teszi a használatát. Ezen kívül vektorok használatát is támogatja, a vektorműveletek már előre definiálva vannak benne, amelyeket az algoritmusok implementálásakor előszeretettel használtam. Egyik funkcionalitása, ami még nagyon sokat segített az algoritmusok implementálásakor, az a translate függvény, amely segítségével a vászon koordinátarendszerének origóját tudjuk eltolni. Ez egy rendkívül hasznos függvény, különböző rotációknál, és olyan alakzatok rajzolásánál, amelyek pozíciója az egérmutató pozicíójától függ. Ezek a műveletek a függvénynek köszönhetően sokkal kevesebb számolással kivitelezhetőek.
\par
A fejlesztés Windows 10 operációs rendszer alatt történt. A programozáshoz, futtatáshoz, teszteléshez a Visual Studio Code fejlesztői környezetet használtam. A szoftver verziókövetése, illetve a fájlok tárolása GitHub repository-n keresztül valósult meg.

\chapter{Az alkalmazás alapfunkciói}
\spacing{1.5}

Ebben a fejezetben az elkészült alkalmazás főbb funkcióit mutatom be, fejtem ki részletesen, külön kitérve a fontosabb, érdekesebb részekre.

\section{Lejátszást vezérlő funkciók}

A fraktálgeneráló algoritmusok futását három gombbal tudjuk vezérelni, ezek a Play, Pause és Start gombok. Mivel az alkalmazást megpróbáltam minél több komponensre bontani, így kellett egy külön szolgáltatás, amelynek a feladata az volt, hogy az egyes eseményeket, amelyeket a felhasználók a gombok megnyomásával váltanak ki, közvetítse annak az algoritmusnak, amely az adott fraktált rajzolja. Ennek köszönhetően az algoritmus tudni fogja mikor kell szünetelni, mikor kell rajzolni, és mikor kell újrakezdeni a rajzolást. A szolgáltatás kódja a következőképpen néz ki:
\begin{lstlisting}[language=typescript]
export class AnimationStateManagerService {
	private state: BehaviorSubject<boolean>;

	constructor() {
		this.state = new BehaviorSubject<boolean>(false);
	}
	
	setState(state: boolean): void {
		this.state.next(state);
	}
	
	getState(): Observable<boolean> {
		return this.state;
	}
}
\end{lstlisting}
A szolgáltatás alapja tehát egy {\it BehaviorSubject} objektum, amely {\it boolean} típusú adatot tud közvetíteni a feliratkozóknak. Amint a felhasználó a lejátszást vezérlő gombok egyikével interakcióba lép, a szolgálatás {\it setState} metódusa hívódik meg, amelynek a paramétere egy igaz vagy hamis értékű változó, attól függően, hogy az algoritmus futása éppen szünetel-e vagy sem. Amikor ez bekövetkezik, akkor a szolgáltatás a paraméterként kapott változó értékét közvetíti a feliratkozóknak, így azok tudni fogják, hogy a felhasználó interakcióba lépett a vezérlőgombok valamelyikével. A feliratkozó ebben az esetben az algoritmus, amely az adott fraktált generálja.

\section{Algoritmus lista és konfigurációs panelek}
A webalkalmazás két oldalsó panellel rendelkezik. Bal oldalon a fraktálokat generáló algoritmusok egy oszlopba rendezett listáját tekinthetjük meg, ezek közül kattintással választhatunk. Az algoritmusok mindegyike rendelkezik előnézettel, tehát aki nem ismerné valamelyik fraktált, láthatja, hogy az vizuálisan hogyan néz is ki. Ezek az előnézetek maguknak a konfigurálható algoritmusoknak egy lebutított változatai, természetesen konfigurációs lehetőségek nélkül, mindegyik paraméter alapértékre van állítva. Minden algoritmus előnézete egy bizonyos szintig iterál, majd ha ezt a szintet elérte, újraindul. 
\par Jobb oldalon a konfigurációs panel található. Ennek a tartalma dinamikus, aszerint generálódik, hogy a kiválasztott algoritmusnak milyen konfigurációs lehetőségei vannak. A konfigurálás háromféle módon történhet:
\begin{itemize}
	\item  csúszka segítségével állíthatunk be értéket az adott algoritmus valamely paraméterének, egy megadott értékkészleten belül
	\item jelölőnégyzet ki-be pipálásával tudjuk jelezni, hogy az adott konfigurációs lehetőséget használni szeretnénk-e
	\item színpaletta segítségével tudjuk a fraktál színét beállítani
\end{itemize}
\par Az algoritmusok listáját egy külön szolgáltatás tárolja. Minden komponens, amelynek szüksége van erre a listára, mint például a fent említett két panel, ettől a szolgáltatástól kéri el. Az algoritmusok listájának modellje a következőképpen néz ki:
\begin{lstlisting}[language=typescript]
export interface IAlgorithmList {
	name: string;
	previewId: string;
	preview: any;
	algorithm: any;
	configurations: IAlgorithmConfiguration[];
	about: string;
}
\end{lstlisting}
Az algoritmusok listájában tárolódik tehát minden algoritmus
\begin{itemize}
	\item neve
	\item előnézetének azonosítója, amely az előnézet megjelenítéséhez szükséges
	\item a fraktál előnézetét kirajzoló algoritmus referenciája
	\item a konfigurációs lehetőségekkel rendelkező algoritmus referenciája
	\item az adott algoritmus konfigurációs lehetőségei, tömbben tárolva
	\item egy pár mondatos ismertető az adott algoritmusról, ez egy külön funkció lételeme amelyről a későbbiekben lesz szó
\end{itemize}
A lista panelnek az algoritmusok előnézetének azonosítójára, és az előnézet algoritmus referenciájára van szüksége, a konfigurációs panelnek pedig az algoritmusok konfigurációs lehetőségeire, amelyek tömbben tárolódnak. Ezen kívül mindkét panel eltüntethető és előhozható egy-egy gomb segítségével, ezzel is növelve az alkalmazás kompaktságát. Az eltüntetés és előhozás animációval történik, amely abból áll, hogy gombnyomásra, CSS szinten tolódik a panelek pozíciója $X$ tengelyen jobbra vagy balra, attól függően, hogy előhozni, vagy eltüntetni szeretnénk a paneleket. Az animáció megvalósítására az Angular Animations API-ját használtam. Ez annyiból áll, hogy a CSS animációkat egy tömbben definiáltam, majd hozzákötöttem az adott panelekhez egy $boolean$ változó segítségével. A $boolean$ változó értéke gombnyomásra változik, így tudni fogja az alkalmazás, hogy eltüntetni, vagy megjeleníteni kell.


\section{Visszatekerés funkció}

A visszatekerés egy sajátos funkciója az alkalmazásomnak, hiszen egy hasonló alkalmazás sem rendelkezik ilyen lehetőséggel. Alapja egy csúszka, ami, ha a felhasználó interakcióba lép vele, az algoritmust visszatekerő módba helyezi. A csúszka kódja az alábbi módon néz ki:
\begin{lstlisting}[language=html]
<div class="slider-container">
<input #slider (click)="rollBack(slider.value)" type="range" min="0" [max]="sliderLength" [value]="sliderLength" class="slider" id="myRange">
</div>
\end{lstlisting}
Lényegében ez egy egyéni stílusú, $range$ típusú beviteli mező, amelyet a $max$ és $value$ direktivák beállításával lehet módosítani. A $value$ paraméter mondja meg, hogy a csúszka éppen hol áll, ez alapértelmezetten a csúszka hosszára van állítva. A $max$ paraméter azt adja meg, hogy a csúszka milyen hosszúságú, ennek az értéke az algoritmus futásával párhuzamosan nő, amit egy $BehaviourSubject$ típusú változó bevezetésével oldottam meg. Amikor az algoritmus egy új iterációját rajzolja ki a fraktálnak, ez a $BehaviorSubject$ objektum egy új értéket közvetít a feliratkozónak, aki jelen esetben a csúszka, így annak a hossza mindig a megfelelő értékre lesz beállítva. Ha a csúszkára kattint a felhasználó, az adott algoritmus visszatekerő módba kerül, és meghívódik a $rollBack$ metódusa. A visszatekeró mód annyit jelent, hogy beállítódnak azok az értékek, amelyeket kiolvasva, az algoritmus tudni fogja meddig kell visszatekerni a fraktál rajzolását. Ennek a forráskódja így néz ki:
\begin{lstlisting}
rollBack(p: any) {
	if (this.rollBack) {
		if (this.play) {
			for (let i = 0; i < this.list.length; i++) {
				this.list[i].draw(p);
			}
			this.rollBack = false;
		}
		else {
			p.background(this.canvasColor);
			for (let i = 0; i < this.rollBackTo; i++) {
				this.list[i].draw(p);
			}
		}
	}
}
\end{lstlisting}
Ez a függvény minden képfrissítéskor újrarajzolja a hátteret, ennek köszönhetően az eddig megrajzolt alakzat törlődik a vászonról. Ezután végigiterál azon a listán, amelyben az algoritmus tárolja a visszatekerés előtt még megrajzolt alakzatokat, és a $rollBackTo$ paraméter segítségével rajzolja vissza azokat az elemeket, amelyek szükségesek. Minden algoritmus rendelkezik ilyen listával, hiszen ez a visszatekerés funkciónak egy fontos alapeleme.


\section{Fraktál leírás funkció}

A webalkalmazásban minden fraktálról egy rövid leírást olvashatunk, amiben szó van az adott fraktálról általánosságban, valamint arról, hogy ki fedezte fel, hol használják, és a jellegzetes tulajdonságairól. Ezt a leírást egy előugró ablak tartalmazza, ami egy globális komponens, minden algoritmus esetén ugyanez az ablak ugrik elő, csupán a tartalma változik dinamikusan. Az ablakot egy gomb megnyomásával hívhatjuk elő. Az ablak HTML kódja így néz ki:
\begin{lstlisting}[language=html]
<div>
	<h1>{{data.title}}</h1>
	<p>{{data.about}}</p>
	<button (click)="close()" mat-button>Bezar</button>
</div>
\end{lstlisting}
Ez egy eléggé rövid kódrészlet, azonban az látszik, hogy a címet és magát a leíró szöveget a $data$ változó tartalmazza. A $data$ változó akkor definiálódik, amikor az ablakot előhozó gombra kattint a felhasználó, értéke pedig az algoritmusokat tartalmazó listából olvasódik ki, attól függően, hogy éppen melyik fraktálgeneráló algoritmus aktív. A gomb megnyomására a következő függvény fut le:
\clearpage
\begin{lstlisting}[language=typescript]
about(): void {
	let dialogRef = this.dialog.open(AboutComponent, {
		width: '800px',
		data: {
			title: this.title,
			about: this.algorithmList[this.activeAlgorithm].about
		}
	});
}
\end{lstlisting}
A $width$ tulajdonság értelemszerűen az előugró ablak szélességét adja meg, ebben az esetben az ablak 800 pixel széles lesz. A magasságot azért nem definiáltam, hogy különböző hosszúságú szövegek esetén se csússzon szét az ablak, a $data$ változóról pedig pár sorral fentebb volt szó.
\chapter{Algoritmusok}
\spacing{1.5}

Ebben a fejezetben azokról a fraktálgeneráló algoritmusokról lesz szó, amelyeket a webalkalmazás részeként implementáltam. Minden algoritmus esetén először egy rövid ismertető lesz olvasható, majd bemutatom az implementáció menetét. Azonban még mielőtt ebbe belekezdenék, szeretnék egy pár fogalmat tisztázni.
\par A Hausdorff-dimenzió a fraktáloknál használt dimenziófogalom, melynek bevezetését az indokolja, hogy egyes alakzatok, idetartozik a fraktálok kerülete, területe, és egyéb dimenzióbeli mértékei ellentmondásos értéket adnak. Például a Sierpiński-szőnyeg lefed egy négyzetet, tehát kétdimenziós, viszont a hézagoknak köszönhetően a területe 0, akárcsak az egydimenziós alakzatoké. A szokásos alakzatok körében a Hausdorff-dimenzió megegyezik az ismert értékekkel: az egyenesé 1, a négyzeté 2, a kockáé 3. Ez abból fakad, hogy a Haussdorff-dimenzió a hosszúság és a terület mérésén alapul. A Hausdorff-dimenzió tehát, a hagyományos, pozitív egész számokkal mérhető dimenziófogalom általánosítása. A Hausdorff-dimenzió nem feltétlenül egész szám~\cite{hausdorff}.
\par Amit még szeretnék egy pár mondatban megfogalmazni az a P5.js könyvtár működési elve. A P5.js-ben történő rajzoláshoz két függvényt kell megvalósítani. Ez a két függvény a $setup$ és a $draw$. A $setup$ a rajzolás elkezdése előtt fut le egyszer, ez tipikusan a vászon létrehozására, háttér, szín, képfrissítés, és egyéb tulajdonságok beállítására használatos. Maga a rajzolás a $draw$ függvényben történik. Ez a függvény minden másodpercben annyiszor fut le amennyi a képfrissítés tulajdonság értéke, ami alapértelmezetten 60-ra van állítva a legtöbb számítógép esetében.


\section{Sierpiński-háromszög}

A Wacław Sierpiński lengyel matematikus által megtalált fraktál úgy áll elő, hogy egy szabályos háromszögből elhagyjuk az oldalfelező pontok összekötésével nyert belső háromszöget, majd az így maradt három háromszögre rekurzívan alkalmazzuk ugyanezt az eljárást. Az így keletkező háromszögek oldalhossza minden lépésben megfeleződik, és területük a negyedére csökken, miközben a középső háromszög eltűnik.\\
A Sierpiński-háromszög konstrukciójához többnyire egyenlő oldalú háromszöget választanak, ez azonban nem kötelező, bármely háromszögből lehet Sierpiński-háromszöget készíteni~\cite{sierp-triangle}.
Az általam megírt algoritmus az alábbi lépések szerint generálja ezt a fraktált:
\begin{itemize}
\item Vegyünk három pontot, melyek a háromszöget határolják be, mindegyik pont a háromszög egy csúcsa
\item Rajzoljunk egy pontot egy tetszőleges helyre, ez lesz a kezdőpont, amelyből az algoritmus indul
\item Véletlenszerűen válasszuk ki a háromszög egyik csúcsát, majd a választott csúcs és a kezdőpont közötti távolság felére rajzoljunk be egy újabb pontot, ez lesz az új kezdőpont
\item Ismételjük meg ezeket a lépéseket
\end{itemize}
A Sierpiński-háromszög Hausdorff-dimenziója: $\dfrac{log(3)}{log(2)} = 1,585$ \cite{sierp-triangle-wiki}.
\clearpage
\begin{figure}[ht]
\begin{center}
	\includegraphics[width=0.5\textwidth]{img/SierpinskiTriangle}
	\caption[labelInTOC]{Sierpiński-háromszög}
\end{center}
\end{figure}
\subsection*{Implementáció}
Minden algoritmus implementációját egy segédosztály megírásával kezdtem. Ezek a segédosztályok azoknak az alakzatoknak felelnek meg, amelyből az adott fraktál gyökéreleme áll, ebben az esetben egy pontnak az osztálya, aminek a neve $Point$. Ezeknek a segédosztályoknak a célja az, hogy az implementációt átláthatóbbá tegyék. A $Point$ osztály egyetlen adattagja egy vektor, amely tárolja egy adott pont $X$ és $Y$ pozícióját a koordináta rendszerben, valamint egy $draw$ függvény, ami kirajzolja az adott pontot. Fontos megemlíteni, hogy a segédosztályban levő $draw$ függvény különbözik a fent említett $draw$ függvénytől, ami a főosztályban kerül implementálásra.
\par Az implementáció további része a főosztályban történt meg. Itt többek között definiálva vannak az algoritmus konfigurációs lehetőségei is, amelyek a következők:
\begin{itemize}
	\item Gyorsaság
	\item Pontvastagság
	\item Véletlenszerű pontvastagság
	\item Pontok közötti távolság
	\item Fixált kezdőpontok
	\item Háromszög részeinek kijelölése
	\item Szín
	\item Szivárvány mód
\end{itemize}
A Sierpiński-háromszög implementációjának legfontosabb részét a $draw$ függvény tartalmazza, ami az alábbi módon néz ki:
\begin{lstlisting}[language=typescript]
p.draw = () => {
	this.setConfigurables(p);
	
	if (this.play) {
		let rand = p.floor(p.random(3));
		if(this.randomStrokeWeight) {
			p.strokeWeight(p.random(0, 10));
		}
	
		if(this.rainbowMode) {
			let h = p.map(p.floor(p.random(this.list.length)), 0, this.list.length, 0, 360);
			p.stroke(h, 255, 255);
		}
		else if (this.customColors && rand == 0) {
			p.stroke(this.color1);
		} 
		else if (this.customColors && rand == 1) {
			p.stroke(this.color2);
		} 
		else if (this.customColors && rand == 2) {
			p.stroke(this.color3);
		} 
	
		let newPoint = p5.Vector.lerp(this.points[rand], 
			this.refPoint, 
			this.lerpValue);
		p.point(newPoint.x, newPoint.y);
		this.refPoint = newPoint;
		
		this.list.push(new Point(newPoint));
		this.rollBackList$.next(this.list);
	}
}	
\end{lstlisting}
Ez egy lerövidített kód, mivel az eredeti túl hosszú, így megpróbáltam csak a lényeget szemléltetni. Látható hogy a rajzolás különböző feltételekhez van kötve, amelyek a konfigurációktól függnek, mint például a véletlenszerű pontvastagság és a szivárvány mód. Továbbá látható, hogy a függvény csak akkor fog rajzolni, ha az algoritmus futása éppen nem szünetel. A $points$ tömb tárolja azt a három pontot, amelyek a háromszöget határolják be, a $refPoint$ pedig azt a pontot, amelyből az algoritmus futása indul. Ez a változó minden új pont kirajzolása esetén értékül kapja ezt az új pontot. A véletlenszerű csúcs a P5.js $random$ metódusa segítségével választódik ki. Az a pozíció, ahová az új pont kerül, a beépített $lerp$ függvény segítségével számítódik ki. A $rollBackList\$$ változó a $draw$ függvény minden lefutása után, vagyis minden új pont kirajzolása után, egy új értéket közvetít a csúszkának, ezzel beállítva annak hosszát.
\section{Sierpiński-szőnyeg}
A Sierpiński-szőnyeg szintén Wacław Sierpiński lengyel matematikus által megtalált fraktál, amely úgy áll elő, hogy egy négyzetet oldalai harmadolásával kilenc kisebb négyzetre bontunk, a középsőt elhagyjuk, és a maradék nyolcon elvégezzük ugyanezt az eljárást (vagyis azoknak is elhagyjuk a közepét), majd az így maradt $8x8$ kisebb négyzeten is, stb. Az eredményül kapott alakzat területe nulla, kerülete végtelen nagy.\\ 
A Sierpiński-szőnyeg Hausdorff-dimenziója: $\dfrac{log(8)}{log(3)} = 1,8928$.
\\A Sierpiński-szőnyeg konstrukciójának lépései:
\begin{itemize}
	\item Vegyünk egy négyzetet
	\item Osszuk fel minden oldalát három részre
	\item A kijelölt pontokat összekötve osszuk fel a négyzetet kilenc kis négyzetre
	\item Töröljük el a középső négyzetet
	\item Ismételjük az előző lépéseket minden kis négyzetre
\end{itemize}
Ezzel az eljárással a négyzet egyre inkább kiürül. Végtelenszer megismételve a Sierpiński-szőnyeg marad~\cite{sierp-carpet-wiki}.
\begin{figure}[!ht]
	\begin{center}
		\includegraphics[width=0.5\textwidth]{img/SierpinskiCarpet}
		\caption[labelInTOC]{A Sierpiński-szőnyeg 4. iterációja}
	\end{center}
\end{figure}
\subsection*{Implementáció}
A Sierpiński-szőnyeghez tartozó segédosztály a $Rectangle$ osztály, amely adattagként tárolja az adott négyzet középpontját, ami egy vektor objektum, és méretét. Mivel ennek a forráskódja túlságosan hosszú, így csak azt a függvényt szemléltetem, amely a felosztást végzi. Ez az alábbi módon néz ki:
\begin{lstlisting}[language=typescript]
divide(): Rectangle[] {
	let rectangles = [];
	
	rectangles.push(new Rectangle(
		new p5.Vector(this.center.x - this.size, this.center.y), 
		this.size / 3));
	rectangles.push(new Rectangle(
		new p5.Vector(this.center.x - this.size, this.center.y + this.size), 
		this.size / 3));
	rectangles.push(new Rectangle(
		new p5.Vector(this.center.x - this.size, this.center.y - this.size), 
		this.size / 3));
	rectangles.push(new Rectangle(
		new p5.Vector(this.center.x + this.size, this.center.y), 
		this.size / 3));
	rectangles.push(new Rectangle(
		new p5.Vector(this.center.x + this.size, this.center.y + this.size), 
		this.size / 3));
	rectangles.push(new Rectangle(
		new p5.Vector(this.center.x + this.size, this.center.y - this.size), 
		this.size / 3));
	rectangles.push(new Rectangle(
		new p5.Vector(this.center.x, this.center.y + this.size), this.size / 3));
	rectangles.push(new Rectangle(
		new p5.Vector(this.center.x, this.center.y - this.size), 
		this.size / 3));
	
	return rectangles;
}
\end{lstlisting}
Látható, hogy ez a függvény egy négyzetből további nyolc kisebb négyzetet készít úgy, hogy az eredeti négyzet középpontját nyolc irányba tolja el, ezek az eltolások lesznek az új négyzetek középpontjai, méretük pedig az eredeti négyzet méretének harmada lesz.
\par Az algoritmus konfigurációs lehetőségei a következők:
\begin{itemize}
	\item Gyorsaság
	\item Kezdő négyzet mérete
	\item Szín
	\item Szivárvány mód
\end{itemize}
Ez az algoritmus konfigurációs lehetőségek terén nem olyan színes, mint például a Sierpinszki-háromszög. Ez annak köszönhető, hogy egyszerűen nincs annyi állítható paraméter, mint az Sierpiński-háromszög esetén. A kevés konfigurációs lehetőség miatt a $draw$ függvény is jóval egyszerűbb a Sierpiński-háromszögétől.
\begin{lstlisting}[language=typescript]
p.draw = () => {
	this.setConfigurables(p);

	if (this.play) {
		let newRectangles: Rectangle[] = [];
		
		for (let i = this.iter; i < this.list.length; i++) {
			if(this.rainbowMode) {
				let h = p.map(i, this.iter, this.list.length, 0, 360);
				p.fill(h, 255, 255);
			}
			this.list[i].draw(p);
			newRectangles = newRectangles.concat(this.list[i].divide());
		}
		
		this.rollBackList$.next(this.list);
		this.iter = this.list.length;
		this.list = this.list.concat(newRectangles);
	}
}
\end{lstlisting}
Végigiterálunk a négyzetek tömbjén, viszont segédváltozók segítségével mindig csak azokon végez műveleteket a ciklus, amelyek még nincsenek kirajzolva, így növelve az algoritmus hatékonyságán. A függvény az adott négyzetet kirajzolja, majd 8 további négyzetet készít belőle a már ismert $divide$ függvény segítségével. Az így kapott elemeket a tömbhöz csatolja, így a következő iterációkor ezeken az új négyzeteken fogja ugyanezt a műveletet elvégezni. A tömb azért tárolja azokat az elemeket is, amelyek már meg lettek jelenítve, mert így visszatekerés esetén ezeket újra ki tudja majd rajzolni a $rollBack$ függvény.

	
\section{Koch-görbe}
A Koch-görbe vagy Koch-hópehely Helge von Koch svéd matematikus által leírt fraktál. A görbét úgy állítjuk elő, hogy veszünk egy szabályos (egyenlő oldalú) háromszöget, minden oldalát megharmadoljuk, és a középső harmadszakaszra újabb szabályos háromszögeket rajzolunk. Majd az így keletkezett háromszögoldalakra újra feltesszük ezt a "kinövést", és ezt a műveletet a végtelenségig folytatjuk. A görbe egyre jobban egy hópehelyhez fog hasonlítani. Természetesen az igazi, teljes hópehely lerajzolása lehetetlen, csupán a hozzávezető állapotok egymásutánját tudjuk ábrázolni. Amint újabb és újabb "kinövéseket" szerkesztünk a háromszögek oldalaira, a hópehely kerülete egyre nő, azaz a hópehely kerülete valójában végtelen. Mivel maga az alakzat megmarad az első háromszög köré írt körének belsejében, így azt mondhatjuk, hogy a területe viszont véges.\\
A Koch-görbe Hausdorff-dimenziója: $\dfrac{log(4)}{log(3)} = 1,26$ \cite{koch-gorbe-wiki}.
\begin{figure}[!ht]
	\begin{center}
		\includegraphics[width=0.75\textwidth]{img/KochCurve}
		\caption[labelInTOC]{Koch-görbe}
	\end{center}
\end{figure}
\subsection*{Implementáció}
A Koch-görbéhez tartozó segédosztály a $Line$ osztály, amely minden egyenes hosszát és két végpontját tárolja. Az osztálynak két fontos metódusa van, ezek az $expandLeft$ és $expandRight$ metódusok. Ezek az adott egyenesre rajzolnak háromszöget úgy, hogy ha háromszög alapja ez az egyenes, akkor az $expandLeft$ metódus adja a háromszög bal oldalát, az $expandRight$ pedig a jobb oldalát.
\clearpage
\begin{lstlisting}
expandLeft(p: any, direction: number, lerp: number, angle: number): Line[] {
	let len = this.length * lerp;
	let alpha = 180 - 2 * p.degrees(angle);
	let sideLength = len * p.sin(angle) / p.sin(p.radians(alpha));
	
	let lerpAmount = (1 - lerp) / 2;
	
	let a = p5.Vector.lerp(this.A, this.B, lerpAmount);
	let dir = p5.Vector.sub(this.B, this.A);
	dir.rotate(-direction * angle);
	let offset = p5.Vector.add(a, dir);
	
	let x = p5.Vector.lerp(a, offset, sideLength / p5.Vector.dist(a, offset));
	
	let newLines = [];
	newLines.push(new Line(this.A, a));
	newLines.push(new Line(a, x));
	
	return newLines;
}
\end{lstlisting}
A metódus paraméterei közé tartozik a $direction$, ami az irányt adja meg, hogy felfelé, vagy lefelé történjen a háromszög oldalának rajzolása, a $lerp$, ami azt mondja meg, hogy a háromszög alapja hányad része az egyenesnek (ez alapértelmezetten az egyenes középső harmada), valamint az $angle$, ami pedig a háromszög alapja és az oldalai között bezárt szöget adja meg. Ezek a paraméterek a konfigurációs panelben állíthatóak. A számolás vektorműveletek segítségével történik, az egyenes két végpontját kivonva egymásból egy új vektort kapunk, amelynek iránya megegyezik az egyenes irányával. Ezt a vektort megfelelő szöggel rotálva, valamint a megfelelő értékkel eltolva megkapjuk a háromszög csúcspontját.
\par A Koch-görbe konfigurációs lehetőségei a következők:
\begin{itemize}
	\item Gyorsaság
	\item Szín
	\item Vonalvastagság
	\item Vonalhosszúság
	\item Szög
	\item Háromszög alapjának mérete (\%)
	\item Fixált kezdővonal
	\item Irány
	\item Szivárvány mód
\end{itemize}
A fixált kezdővonal opció kikapcsolásával testreszabhatjuk a gyökér helyét, és méretét, valamint az egér görgőjével rotálhatjuk, majd kattintással elhelyezhetjük a vászonon. A gyökér megadása esetén az egér mozgására a változások valós időben láthatóak. Kattintáskor az alábbi függvény fut le:
\begin{lstlisting}
p.handleMousePressed = () => {
	if (this.play && !this.useFixedRoot && this.customRoot == null) {
		let center = p.createVector(p.mouseX, p.mouseY);
		let x = p.createVector(p.mouseX - this.length / 2, p.mouseY);
		let y = p.createVector(p.mouseX + this.length / 2, p.mouseY);
		
		let xDir = p5.Vector.sub(x, center);
		xDir.rotate(this.rotation);
		
		let yDir = p5.Vector.sub(y, center);
		yDir.rotate(this.rotation);
		
		let xOffset = p5.Vector.add(center, xDir);
		let yOffset = p5.Vector.add(center, yDir);
		
		this.customRoot = new Line(xOffset, yOffset);
		this.root = this.customRoot;
		this.lines = [this.root];
	}
}
\end{lstlisting}
Kattintáskor az egyenes úgy rajzolódik ki, hogy annak középpontja megegyezik a kurzor pozíciójával. Az egyenes bal oldali végpontját úgy kapjuk meg, hogy a kurzor $X$ pozíciójából kivonjuk az egyenes hosszának a felét. Hasonlóan a jobb oldali végpont esetében, annyi különbséggel, hogy ott kivonás helyett hozzáadjuk. Az így kapott egyenes vízszintes, ezért, ha a felhasználó az egér görgője segítségével rotálta, további számolásokra van szükség. Mindkét végpontból kivonjuk az egyenes középpontját, így két megfelelő irányba mutató vektort kapunk, amit a vektorok beépített $rotate$ függvényével könnyen megfelelő pozícióba rotálhatunk. A $this.rotation$ változó tárolja azt az értéket, amennyivel a felhasználó az egyenest rotálta, ezt az értéket kapja paraméterül a $rotate$ függvény. A Koch-görbe $draw$ függvénye nagyon hasonló, mint a Sierpiński-szőnyegé, így ezt nem szemléltetném.
\section{Lévy C-görbe}
A Lévy C-görbét először Ernesto Cesàro és Georg Faber írta le és tanulmányozta a differenciálhatóságát, azonban mégis Paul Lévy francia matematikus nevét viseli, aki a fraktál önhasonlóságát definiálta~\cite{levy-c-wiki}. Ez a fraktál alapvetően nagyon hasonlít a Koch-görbére, egyedüli eltérés, hogy az egyenesekre való háromszögek rajzolásakor nem az egyenes egy bizonyos része adja a háromszög alapját, hanem a teljes egyenes. Mivel az algoritmus implementációja és a konfigurációs lehetőségek is nagyon hasonlítanak a Koch-görbe algoritmuséhoz, így ezt a fraktált nem részletezem tovább.
\begin{figure}[!ht]
	\begin{center}
		\includegraphics[width=0.5\textwidth]{img/LevyCCurve}
		\caption[labelInTOC]{Lévy C-görbe}
	\end{center}
\end{figure}
\section{Hilbert-görbe}
A Hilbert-görbe a térkitöltő fraktálok csoportjába tartozik. Az első térkitöltő görbét ugyan Peano alkotta meg, de Hilbert volt az, aki először érthető geometriai képet tudott mutatni egy általa definiált térkitöltő görbéről. Megadott egy geometriai generáló eljárást, amivel létrehozta ezen görbék egy osztályát~\cite{hilbert-wiki}. 
\par A Hilbert-görbe első iterációjának megrajzolása úgy történik, hogy veszünk egy egységnégyzetet a következő pontokkal:
\begin{itemize}
	\item $(0, 0)$ pont a négyzet bal alsó sarka
	\item $(0, 1)$ pont a négyzet bal felső sarka
	\item $(1, 1)$ pont a négyzet jobb felső sarka
	\item $(1, 0)$ pont a négyzet jobb alsó sarka
\end{itemize}
Ezt az egységnégyzetet további 4 szabályos négyzetre bontjuk, és egy töröttvonalat húzunk, úgy, hogy az átmenjen mind a négy négyzet középpontján az alábbi sorrendet követve: A bal alsó sarokban levő négyzet középpontjából indul, majd felfelé megy a bal felső négyzet középpontjáig, innen jobbra a jobb felső négyzet középpontjáig, majd a jobb alsó négyzet középpontjában áll meg. Ezzel egy fordított U alakzatot kapunk. 
\begin{figure}[!ht]
	\centering
	\begin{subfigure}{.5\textwidth}
		\centering
		\includegraphics[width=.4\linewidth]{img/HilbertCurve1-1}
	\end{subfigure}%
	\begin{subfigure}{.5\textwidth}
		\centering
		\includegraphics[width=.4\linewidth]{img/HilbertCurve1-2}
	\end{subfigure}
	\caption{A Hilbert-görbe 1. iterációja}
	\label{fig:test}
\end{figure}
\\A Hilbert-görbe második iterációja 4 egységnégyzetből áll, mindegyikben megrajzolva az első iteráció fordított U alakzatát. A bal alsó négyzetet 90 fokkal jobbra, a jobb alsó négyzetet 90 fokkal balra forgatjuk, így a két alsó U alakzat egymástól "elfelé" néz. Végül az U alakzat megrajzolásával megegyező sorrendben (bal alsó, bal felső, jobb felső, jobb alsó) összekötjük mind a 4 egységnégyzetben levő alakzatok egymáshoz legközelebbre eső végpontjait.
\begin{figure}[!ht]
	\begin{center}
	\includegraphics[width=0.5\textwidth]{img/HilbertCurve2-1}
	\caption[labelInTOC]{Hilbert görbe 2. iterációja}
\end{center}
\end{figure}


\subsection*{Implementáció}
Az implementáció megvalósításához elengedhetetlen két képlet használata, ezek a következők:\\
$N = 2^I$,
$P = N^2$, ahol $N$ a négyzetek száma, $I$ az iteráció, $P$ pedig a pontok száma. Az iteráció adott, ugyanis a felhasználó konfigurációs lehetőségként meg kell adja, hogy hanyadik iterációig szeretné, hogy az algoritmus fusson. 
Az algoritmus egy $setup$ függvénnyel indul, ez különbözik a P5.js $setup$ függvényétől, és az alábbi módon néz ki:
\begin{lstlisting}[language=typescript]
 setup(): void {
	 this.squares = Math.pow(2, this.order);
	 this.points = Math.pow(this.squares, 2);
	 this.length = Math.min(this.width, this.height) / this.squares;
	 for (let i = 0; i < this.points; i++) {
		 this.path[i] = this.hilbert(i);
		 this.path[i].mult(this.length);
		 this.path[i].add(this.length / 2, this.length / 2);
	 }
	 this.root = new Line(new p5.Vector(this.path[0].x, this.path[0].y),
	 new p5.Vector(this.path[1].x, this.path[1].y));
 }
\end{lstlisting}
A függvény először, a fentebb említett képletek segítségével kiszámolja, hogy az adott iteráció esetén hány négyzetből, és hány pontból fog állni a Hilbert-görbe. Ezután kiszámolja, hogy mekkora lesz egy-egy vonal hossza. Mivel a vászon szélessége és hosszúsága különbözik, ezért a függvény a kettő közül a kisebbiket veszi számításba, így a görbe biztosan bele fog férni a vászonba, majd ezt leosztja a négyzetek számával. Egy $for$ ciklus segítségével a függvény 0-tól a pontok számáig iterál, és minden iterációban meghívja a $hilbert$ függvényt, amelynek paraméterül a ciklusváltozót adja. A $hilbert$ függvény egy vektort ad vissza, amely megmondja, hogy az adott index, amit paraméterként átadtunk neki, az U alakzat mely pontjával egyezik meg. Ez $(0, 0)$ vektor esetén az alakzat bal felső pontja, $(0, 1)$ esetén a bal alsó pontja, és így tovább... Ezeket a vektorokat egy tömbhöz adja, megszorozza a fentebb kiszámított hosszúsággal, hogy a vektorok (U pontjai) közötti távolság megfelelő legyen, majd a vektor $X$ és $Y$ értékeihez hozzáadja a vászon hosszának vagy szélességének felét, amelyik éppen kisebb, ezzel megfelelő helyre pozícionálva őket.
\par Az algoritmus $draw$ függvénye a $setup$ függvény által feltöltött tömbön iterál végig úgy, hogy minden képfrissítéskor eggyel mélyebbre megy a pontok tömbjén, és egy vonallal összeköti az adott pontot a tömbben levő előző ponttal, ezzel egy animációs hatást keltve. 
\par Az algoritmus viszonylag kevés konfigurációs lehetőséggel bír, ezek a következők:
\begin{itemize}
	\item Gyorsaság
	\item Vonalvastagság
	\item Iteráció
	\item Szín
	\item Szivárvány mód
\end{itemize}
\section{Pitagorasz-fa}
A Pitagorasz-fa egy négyzetekből álló fraktál, amelyet Albert Bosman fedezett fel. A nevét onnan kapta, hogy minden egymást érintő négyzet hármas egy szabályos háromszöget zár be. Ha a legnagyobb négyzet $L x L$ méretű (törzs), akkor a teljes Pitagorasz-fa elfér egy $6L x 4L$ méretű négyzetben~\cite{pitagorasz-fa}. A hagyományos, egyenlő szárú Pitagorasz-fa a következőképpen konstruálható: az első iterációban létrejön a törzse, amely egy négyzet. A második iterációban a törzsnek a felső élére egy egyenlő szárú derékszögű háromszöget rajzolunk úgy, hogy átfogója a négyzet felső éle, valamint a háromszög két befogójából kiágazik az első két ág, amelyek szintén négyzetek. Ezután minden iterációban ez ismétlődik, azaz minden korábbi négyzet felső élére egy egyenlő szárú derékszögű háromszög nő, és azok befogói új négyzetágakat növesztenek. Minden négyzet mérete $\dfrac{\sqrt{2}}{2}$ értékkel skálázódik le a szülő méretéhez képest.
\begin{figure}[!ht]
	\begin{center}
		\includegraphics[width=0.5\textwidth]{img/PythagorasTree}
		\caption[labelInTOC]{Pitagorasz-fa}
	\end{center}
\end{figure}

\subsection*{Implementáció}
A kód átláthatósága érdekében itt is egy segédosztályt írtam először, a $Rectangle$ osztályt. Ebben az osztályban tárolódnak a négyzetek pontjai, illetve mérete. Továbbá két fontos függvényt tartalmaz, az $expandLeft$ és $expandRight$ függvényeket, amelyek az adott négyzet bal és jobb oldali ágát adják meg. 
\begin{lstlisting}[language=typescript]
expandRight(p: any, angle: number): Rectangle {
	let A: p5.Vector;
	let B: p5.Vector;
	let C: p5.Vector;
	let D: p5.Vector = this.B;
	
	let center = p5.Vector.lerp(this.A, this.B, .5);
	let dir = p5.Vector.sub(this.A, center);
	dir.rotate(angle);
	let offset = p5.Vector.add(center, dir);
	
	C = p5.Vector.lerp(center, offset, this.size * 0.5 / p5.Vector.dist(offset, center));
	
	A = p5.Vector.sub(D, C);
	A.rotate(-p.PI/2);
	A = p5.Vector.add(C, A);
	A = p5.Vector.lerp(C, A, 1);
	
	B = p5.Vector.sub(C, D);
	B.rotate(p.PI/2);
	B = p5.Vector.add(D, B);
	B = p5.Vector.lerp(D, B, 1);
	
	let left = new Rectangle(A, B, C, D);
	
	return left;
}
\end{lstlisting}
Az új négyzetek pontjai vektorműveletek sorozatával számolódnak ki. A pontok elnevezése a következő rendszert követi: 
\begin{itemize}
	\item $A$ - bal felső
	\item $B$ - jobb felső
	\item $C$ - bal alsó
	\item $D$ - jobb alsó
\end{itemize}
A jobb oldali ág négyzetének $D$ pontja megegyezik a szülőnégyzet $B$ pontjával. A $C$ pontot úgy kapjuk meg, hogy a szülőnégyzet felső éléből egy megfelelő irányba mutató vektort hozunk létre, amelyet rotálunk egy bizonyos szöggel. A rotálási szöget a függvény paraméterként kapja, ugyanis ez az érték testreszabható a konfigurációs panelben. A maradék $A$ és $B$ pontok már könnyen kiszámolhatóak, ezeket úgy kapjuk meg, hogy két új vektort hozunk létre, egyik a $C$-ből $D$-be, másik a $D$-ből $C$-be mutat, majd ezeket 90 fokkal rotáljuk a megfelelő irányba. A függvény visszatérési értékként egy új négyzetet hoz létre a kiszámított pontokból. Az $expandLeft$ függvény hasonlóan működik, csupán a rotálási irányok változnak. 
\par Az algoritmus konfigurációs lehetőségei a következők:
\begin{itemize}
	\item Gyorsaság
	\item Fixált szög
	\item Fixált gyökér
	\item Oldalhosszúság
	\item Szín
	\item Szivárvány mód
\end{itemize}
Ha a fixált gyökér lehetőséget kikapcsoljuk, akkor tetszőleges helyen helyezhetünk el gyökérnégyzetet a vászonon. Ezt az egér görgője segítségével rotálhatjuk, illetve az oldalhosszúság opció értékének változtatásával megadhatjuk a méretét. Hasonlóan, a fixált szög lehetőség kikapcsolásával tetszőleges szöggel elhajlíthatjuk az ágakat jobb vagy bal irányba. A szög megadása az egér segítségével történik. Ahhoz, hogy a felhasználóknak érthető legyen a szög megadásának módja, a gyökér felső élének középpontjából egy egyenes indul az egérmutató pozíciója felé. Ennek az egyenesnek a hossza az oldalhosszúság felével egyezik meg. Kattintáskor a program kiszámolja az egyenes és a négyzet felső éle között bezárt szöget, és ezt felhasználja az ágak generálásakor.
\section{Fraktál-fa}
A Fraktál-fa egy viszonylag egyszerűen megkonstruálható fraktál, ami a legismertebb fraktálok közé tartozik. Szerkezete nagyon hasonlít a Pitagorasz-fáéhoz, annyi különbséggel, hogy négyzetek helyett egyenesekből tevődik össze. A Fraktál-fa első iterációjában csak a törzs van, a másodikban a törzsből két ág nő ki egy bizonyos szöget bezárva a törzzsel. Ez a szög tetszőlegesen állítható a konfigurációs panelben. A további iterációkban ez ismétlődik, tehát a már meglévő ágakból új ágak nőnek ki, ez a végtelenségig ismételhető.  
\begin{figure}[!ht]
	\begin{center}
		\includegraphics[width=0.5\textwidth]{img/FractalTree}
		\caption[labelInTOC]{Fraktál fa}
	\end{center}
\end{figure}
\subsection*{Implementáció}
A szerkezeti hasonlóságokból eredően az algoritmus implementációja sok helyen hasonlít a Pitagorasz-fáéhoz. Itt is egy segédosztály megírásával kezdtem az implementációt. Ez az osztály tárolja az egyenesek $A$ és $B$ végpontjait, valamint hosszát. Ezen kívül még tartalmaz egy $branch$ nevezetű függvényt, amely egy adott vonalra meghívva megadja annak az ágait. 
\begin{lstlisting}
branch(p: any, angleLeft: number, angleRight: number, lerpPercentage: number): Line[] {
	let lines: Line[] = [];
	let lerpAmount = this.length * lerpPercentage / this.length;
	
	let dir = p5.Vector.sub(this.A, this.B);
	let xRotated = dir.rotate(angleLeft);
	let xOffset = p5.Vector.add(this.A, xRotated);
	let yRotated = dir.rotate(-angleLeft - angleRight);
	let yOffset = p5.Vector.add(this.A, yRotated);
	let x = p5.Vector.lerp(this.A, xOffset, lerpAmount);
	let y = p5.Vector.lerp(this.A, yOffset, lerpAmount);
	
	lines.push(new Line(x, this.A));
	lines.push(new Line(y, this.A));
	
	return lines;
}
\end{lstlisting}
Ennek a függvénynek 3 fontos paramétere van, az $angleLeft$, ami megadja a bal oldali ág és a szülő által bezárt szöget, az $angleRight$, ami a jobb oldali ág és a szülő által bezárt szöget adja meg, valamint a $lerpPercentage$, ami azt adja meg, hogy az új ágak hossza hány százaléka a szülőág hosszának. Mivel az egyenesek $A$ és $B$ végpontjait vektorokként tárolja a segédosztály, ezért az új ágak végpontjai vektorműveletek segítségével könnyen kiszámolhatóak. Az $A$ végpontból kivonva a $B$ végpontot, egy, a szülőág irányával megegyező irányú vektort kapunk. Ezt, a paraméterben kapott értékekkel jobbra és balra forgatva, megkapjuk az új ágak $A$ végpontjait. Az új ágak $B$ végpontjai a szülőág $A$ végpontjával egyeznek meg értelemszerűen. A visszatérési értéke egy tömb, amely az új ágakat tartalmazza.
\par Az algoritmus konfigurációs lehetőségei a következők:
\begin{itemize}
	\item Gyorsaság
	\item Vonalvastagság
	\item Vonalhosszúság
	\item Ágak száma
	\item Elforgatási szög
	\item Ág mérete (\%)
	\item Véletlenszerű szög
	\item Fixált kezdővonal
	\item Szín
	\item Szivárvány mód
\end{itemize}
Az ágak száma opcióval megadható, hogy az egyes ágakból kettő vagy három új ág keletkezzen. Három ág esetén a középső ág iránya megegyezik a szülőág irányával. A fixált kezdővonal opció kikapcsolásával a fa törzsét adhatjuk meg tetszőlegesen, hasonlóan, mint az előző algoritmusoknál.
\section{H-fa}
A H-fa struktúrája hasonló a Fraktál-fáéhoz. Ez a fraktál merőleges egyenesek közvetlen egymás mellé helyezésével konstruálható meg. Minden egyenes mindkét végpontján egy rá merőleges egyenes halad át, amelynek hossza mindig $\sqrt{2}$-vel kisebb az előző egyenes hosszától. Ez egy alapértelmezett érték, ami a konfigurációs panelben tetszőlegesen állítható. A fraktál a nevét onnan kapta, hogy a benne ismétlődő minta a H betűre emlékeztet.\\ 
A H-fa Hausdorff dimenziója $2$.
\begin{figure}[!ht]
	\begin{center}
		\includegraphics[width=0.5\textwidth]{img/HTree}
		\caption[labelInTOC]{H-fa}
	\end{center}
\end{figure}
\subsection*{Implementáció}
A H-fa egy viszonlyag egyszerűnek mondható fraktál, ezért az implementációja is egyszerű. A függvények, amelyek az új ágak pontjainak kiszámolását végzik itt is egy segédosztályban vannak megírva, ezek az $expandLeft$ és $expandRight$ függvények.
\begin{lstlisting}
expandLeft(p: any, lerp: number) {
	let dir = p5.Vector.sub(this.A, this.B);
	dir.rotate(p.PI / 2);
	let xOffset = p5.Vector.add(this.A, dir)
	let x = p5.Vector.lerp(this.A, xOffset, lerp / 2);
	
	dir.rotate(-p.PI);
	let yOffset = p5.Vector.add(this.A, dir);
	let y = p5.Vector.lerp(this.A, yOffset, lerp / 2);
	
	return new Line(x, y);
}
\end{lstlisting}
A segédosztályban az egyenesek végpontjai vektorokként vannak tárolva. A végpontokat egymásból kivonva egy vektort kapunk, amelynek iránya megegyezik az egyenes irányával, majd ezt jobbra és balra forgatva megkapjuk a bal oldali ág két végpontját. Az eredeti egyenes bal oldali végpontjából az új végpontok irányába csak az új ág hosszának felével megegyező mértékig megyünk, ezzel megkapva az új ág megfelelő hosszát. A jobb oldali ág kiszámítása hasonlóan történik.
\par Az algoritmushoz tartozó konfigurációs lehetőségek a következők:
\begin{itemize}
	\item Gyorsaság
	\item Vonalvastagság
	\item Vonalhosszúság
	\item Ág hosszúság (\%)
	\item Fixált kezdővonal
	\item Szín
	\item Szivárvány mód
\end{itemize}
\chapter*{Nyilatkozat}
%Egy üres sort adunk a tartalomjegyzékhez:
\addtocontents{toc}{\ }
\addcontentsline{toc}{section}{Nyilatkozat}
%\hspace{\parindent}

% A nyilatkozat szövege más titkos és nem titkos dolgozatok esetében.
% Csak az egyik tipusú myilatokzatnak kell a dolgozatban szerepelni
% A ponok helyére az adatok értelemszerûen behelyettesídendõk es
% a szakdolgozat /diplomamunka szo megfeleloen kivalasztando.


%A nyilatkozat szövege TITKOSNAK NEM MINÕSÍTETT dolgozatban a következõ:
%A pontokkal jelölt szövegrészek értelemszerûen a szövegszerkesztõben és
%nem kézzel helyettesítendõk:

\noindent
Alulírott \makebox[4cm]{\dotfill} szakos hallgató, kijelentem, hogy a dolgozatomat a Szegedi Tudományegyetem, Informatikai Intézet \makebox[4cm]{\dotfill} Tanszékén készítettem, \makebox[4cm]{\dotfill} diploma megszerzése érdekében.

Kijelentem, hogy a dolgozatot más szakon korábban nem védtem meg, saját munkám eredménye, és csak a hivatkozott forrásokat (szakirodalom, eszközök, stb.) használtam fel.

Tudomásul veszem, hogy szakdolgozatomat / diplomamunkámat a Szegedi Tudományegyetem Informatikai Intézet könyvtárában, a helyben olvasható könyvek között helyezik el.

\vspace*{4cm}

\begin{tabular}{lc}
Szeged, \today\
\hspace{2cm} & \makebox[6cm]{\dotfill} \\
& aláírás \\
\end{tabular}


\vspace*{2cm}

%A nyilatkozat szövege TITKOSNAK MINÕSÍTETT dolgozatban a következõ:

%\noindent
%Alulírott \makebox[4cm]{\dotfill} szakos hallgató, kijelentem, hogy a dolgozatomat a Szegedi Tudományegyetem, Informatikai Intézet \makebox[4cm]{\dotfill} Tanszékén készítettem, \makebox[4cm]{\dotfill} diploma megszerzése érdekében.

%Kijelentem, hogy a dolgozatot más szakon korábban nem védtem meg, saját munkám eredménye, és csak a hivatkozott forrásokat (szakirodalom, eszközök, stb.) használtam fel.

%Tudomásul veszem, hogy szakdolgozatomat / diplomamunkámat a TVSZ 4. sz. mellékletében leírtak szerint kezelik.

%\vspace*{1cm}

%\begin{tabular}{lc}
%Szeged, \today\
%\hspace{2cm} & \makebox[6cm]{\dotfill} \\
%& aláírás \\
%\end{tabular}

\chapter*{Köszönetnyilvánítás}
\addcontentsline{toc}{section}{Köszönetnyilvánítás}
Köszönöm a figyelmet.





%Irodalomjegyzek ha kell
%\bibliography{mybib}
%\bibliographystyle{plain}



\end{document}