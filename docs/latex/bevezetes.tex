\chapter*{Bevezetés}
\addcontentsline{toc}{section}{Bevezetés}
\spacing{1.5}
\section*{A fraktálokról átalánosságban}
A fraktálok egy viszonylag újnak nevezhető terület a matematikában, és habár komolyabban vizsgálni őket csak a számítógépek megjelenése után kezdték el igazán, már a XX. század előtt is foglalkoztak a matematikusok ezekkel a különös geometriai alakzatokkal. Az elnevezést 1975-ben Benoît Mandelbrot adta, a latin fractus (vagyis törött, törés) szó alapján, ami az ilyen alakzatok tört számú dimenziójára utal, bár nem minden fraktál tört dimenziós (ilyenek például a síkkitöltő görbék). \cite{fraktal}
Tört dimenziósnak nevezik azokat az alakzatokat, amelyek nincs területük, többek mint egy egyenes, viszont kevesebbek mint egy síkidom. Nagyszerű példa erre a Sierpiński-szőnyeg. Mandelbrot {\it The Fractal Geometry of Nature (A Természet Fraktálgeometriája)} című munkája mutatta be, és magyarázta el először a fogalmakat, melyek alapjául szolgálnak ennek az új területnek. \cite{fraktal-wiki}
\par A fraktálok önhasonló, végtelenül komplex matematikai alakzatok, melyek változatos formáiban legalább egy felismerhető ismétlődés tapasztalható. Az önhasonlóság azt jelenti, hogy egy kisebb rész felnagyítva ugyanolyan struktúrát mutat, mint egy nagyobb rész. Ilyen bizonyos léptékig például a természetben a villám mintázata, a levél erezete, a hópelyhek alakja, a fa ágai. A fraktál szóval rendszerint az önhasonló alakzatok közül azokra utalnak, amelyeket egy matematikai formulával le lehet írni, vagy meg lehet alkotni. \cite{fraktal} A fraktálok egy másik tulajdonsága, hogy sehol sem differenciálhatók. Ennek oka az, hogy bár a fraktálok folytonosak, végtelenül gyűröttek. A fraktálokat úgy írhatjuk le, hogy megvizsgáljuk, hogyan változnak különböző felbontásoknál. 
\section*{A fraktálok felhasználási területei}
A fraktálok gyakorlati felhasználásának köre folyamatosan bővül. Rengeteg helyen találkozhatunk velük a művészetektől az orvosláson át a számítástechnikáig. Ma már bizonyított tény, hogy fraktálok, illetve fraktálszerű alakzatok a természetben is gyakran előfordulnak. Megfigyelések alapján tudjuk, hogy ezek az alakzatok egy meghatározott növekedési struktúrát követnek. Ilyenek a fák, kristályok, vagy a felhők. Számtalan művész használta a fraktálok ábrázolásának technikáját, annak ellenére, hogy pontos definícióját, matematikai hátterét nem ismerték.
Olyan neveket érdemes itt megemlíteni, mint például Vincent Van Gogh, vagy akár Maurits Cornelis Escher. A építészetben is sok önismétlődő részletet találhatunk gondoljunk csak a gótikus és barokk épületek ismétlődő támpilléreire, oszlopsoraira. \cite{alkalmazas}
\par A fraktálok kiválóan alkalmazhatóak képi és hanganyag tömörítésére, feldolgozására. Ha egy hangot vagy képet fraktálokra bontunk, utána az adott darabot leíró algoritmussal és paramétereivel könnyen összehasonlíthatóvá válnak. Továbbá az algoritmusok és a hozzá tartozó paraméterek letárolása kevesebb területet vesz igénybe, mintha a nyers adatokat tárolnánk le. \cite{alkalmazas}