\chapter*{Összegzés}
\addcontentsline{toc}{section}{Összegzés}
\spacing{1.5}
Szakdolgozatom többnyire a fraktál bemutató keretrendszer implementálásának menetét mutatta be. Ez egy olyan keretrendszer amely a felhasználóknak lehetőséget ad arra, hogy nyolc féle fraktált generáljanak, a generáló algoritmusokat konfigurálják, az eredményt képként exportálják, a generálás közben visszalépjenek valamely előző lépésre, valamint mind a nyolc keretrendszerben szereplő fraktálról egy rövid bemutatót olvassanak.
\par A szakdolgozat elején az olvasók a fraktálokról egy bemutatót olvashatnak. Szó van a fraktálok tulajdonságairól, alkalmazási területeiről, nevének eredetéről. Ez a fejezet lényegében a fraktálok mivoltáról szól. A fraktálok röviden fogalmazva olyan alakzatok, amelyek szerkezetében legalább egy felismerhető ismétlődés található. A fraktálok legfontosabb tulajdonsága az önhasonlóság.
\par Az ezt követő szekcióban az interneten található hasonló webalkalmazásokról van szó röviden, ismertetem a felhasznált technológiákat, amelyek az Angular keretrendszer, és a p5.js, ami egy vászon rajzolásra alkalmas JavaScript könyvtár. 
\par A további fejezetek témáját maga a keretrendszer tölti ki. Bemutatásra kerülnek az alkalmazás főbb funkciói, amelyek többek között a konfigurálás, exportálás, és visszatekerés. Részletezem mindegyik fraktál generáló algoritmus implementációjának menetét, valamint az alkalmazásban szereplő minden fraktálról egy ismertetőt olvashatnak. A keretrendszer részeként implementált nyolc fraktál generáló algoritmus a következő: 
\begin{itemize}
	\item Sierpiński- háromszög
	\item Sierpiński-szőnyeg
	\item Fraktál-fa
	\item Pitagorasz-fa
	\item Lévy C-görbe
	\item Koch-görbe
	\item Hilbert-görbe
	\item H-fa
\end{itemize}
\par Az elkészült alkalmazást a jövőben bővíteni szeretném. A már megírt algoritmusokhoz további konfigurációs lehetőségeket tervezek implementálni, valamint a magát keretrendszert kibővíteni még néhány fraktál algoritmussal, mint például a Sárkány-görbe, vagy a Sam-görbe.